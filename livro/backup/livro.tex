%% abtex2-modelo-livro.tex, v-1.9.6 ycherem
%% Copyright 2012-2016 by abnTeX2 group at http://www.abntex.net.br/
%%
%% This work may be distributed and/or modified under the
%% conditions of the LaTeX Project Public License, either version 1.3
%% of this license or (at your option) any later version.
%% The latest version of this license is in
%%   http://www.latex-project.org/lppl.txt
%% and version 1.3 or later is part of all distributions of LaTeX
%% version 2005/12/01 or later.
%%
%% This work has the LPPL maintenance status `maintained'.
%%
%% Further information is available on 
%% http://www.abntex.net.br/
%%
%% This work consists of the files
%% abntex2-modelo-livro.tex, abntex2-modelo-references.bib,   
%% abntex2-modelo-livro-pintassilgo.jpg,
%% abntex2-modelo-livro-saira-amarela.jpg,
%% abntex2-modelo-livro-bandeirinha.jpg
%%

\documentclass[
	% -- opções da classe memoir --
	10pt,				% tamanho da fonte
	openright,			% capítulos começam em pág ímpar (insere página vazia caso preciso)
	twoside,			% para impressão em recto e verso. Oposto a oneside
	a5paper,			% tamanho do papel. 
	% -- opções da classe abntex2 --
	%chapter=TITLE,		% títulos de capítulos convertidos em letras maiúsculas
	%section=TITLE,		% títulos de seções convertidos em letras maiúsculas
	%subsection=TITLE,	% títulos de subseções convertidos em letras maiúsculas
	%subsubsection=TITLE,% títulos de subsubseções convertidos em letras maiúsculas
	% -- opções do pacote babel --
	english,			% idioma adicional para hifenização
	french,				% idioma adicional para hifenização
	spanish,			% idioma adicional para hifenização
	brazil,				% o último idioma é o principal do documento
	sumario=tradicional
]{abntex2}

% compilação de fontes

\usepackage{ifxetex}
\ifxetex
  % % se for utilizar as fontes do sistema: **escolha sua fonte**
  \usepackage{polyglossia}
  \setmainlanguage{brazil}
  \setotherlanguages{french,english,spanish,german,italian}
  \usepackage{fontspec}
  \defaultfontfeatures{Ligatures=TeX}
  % comandos de fontes
  \setmainfont[Numbers=OldStyle]{Minion Pro} %fonte principal (serifada)
  \setsansfont[Scale=0.9]{Myriad Pro} %fonte sem serifas
  \setmonofont[Scale=MatchLowercase]{Consolas} % fonte monoespaçada
\else
  % % se for utilizar pdflatex
  \usepackage[utf8]{inputenc}
  \usepackage[T1]{fontenc}
  \usepackage{fourier}
  \usepackage[defaultsans]{droidsans} %fonte droid sans como default sans, ao invés de CM Sans.
  \usepackage[scaled=0.9]{inconsolata} %fonte inconsolata para códigos
  \usepackage[defaultmono,scale=0.8]{droidmono} %fonte droid mono para códigos
\fi

%% Observação: o pacote polyglossia pode apresentar erro ao ser utilizado com ifxetex + babel. 
%% Se isso acontecer, atualize o pacote para a versão mais recente ou utilize somente uma das sequências (pdflatex ou xelatex), comentando ou apagando a outra.

\usepackage{microtype} 				% para melhorias de justificação
\usepackage[dvipsnames]{xcolor} 	% para cores
\usepackage{graphicx} 				% para imagens
\usepackage{booktabs,tabularx,rotating}% para tabelas
\usepackage{mdframed} 				% para caixas de texto como na CIP do verso do título
\usepackage{multicol}				% tabelas com colunas mescladas
\usepackage{lettrine}				% letras capitulares
\usepackage{xspace} 				% para nao precisar de espaços com {} depois de comandos
									% como \LaTeX e abreviações criadas pelo usuário
\usepackage{lipsum} 				% para texto de preenchimento de exemplo
\usepackage{leading}				% espaçamento entrelinhas (leading)
\leading{13pt}

% ---
% Pacotes de citações
% ---
\usepackage[brazilian,hyperpageref]{backref}	 % Paginas com as citações na bibl
\usepackage[alf]{abntex2cite}	% Citações padrão ABNT

% ---
% Inclusão de códigos de LP
% ---
\usepackage{listings}
\lstset{
  language=C++,
  basicstyle=\ttfamily\small, 
  keywordstyle=\color{blue}, 
  stringstyle=\color{verde}, 
  commentstyle=\color{red}, 
  extendedchars=true, 
  showspaces=false, 
  showstringspaces=false, 
  numbers=left,
  numberstyle=\tiny,
  breaklines=true, 
  backgroundcolor=\color{green!10},
  breakautoindent=true, 
  captionpos=b,
  xleftmargin=0pt,
}
% ---
% Configurações do pacote backref
% Usado sem a opção hyperpageref de backref
\renewcommand{\backrefpagesname}{Citado na(s) página(s):~}
% Texto padrão antes do número das páginas
\renewcommand{\backref}{}
% Define os textos da citação
\renewcommand*{\backrefalt}[4]{
	\ifcase #1 %
		Nenhuma citação no texto.%
	\or
		Citado na página #2.%
	\else
		Citado #1 vezes nas páginas #2.%
	\fi}%
% ---

% ---
% Informações do documento
% ---
\titulo{ESP8266 - Uma Introdução ao IoT}
\autor{Gabriel M. de Melo}
\data{2017, v-1.0}
\preambulo{Breve sinopse do livro}
\local{Lavras-MG}
\instituicao{Universidade Federal de Lavras}

% alterando o aspecto da cor azul
\definecolor{blue}{RGB}{41,5,195}

% informações do PDF
\makeatletter
\hypersetup{
     	%pagebackref=true,
		pdftitle={\@title}, 
		pdfauthor={\@author},
    	pdfsubject={\imprimirpreambulo},
	    pdfcreator={LaTeX with abnTeX2},
		pdfkeywords={abnt}{latex}{abntex}{abntex2}{livro}, 
		colorlinks=true,       		% false: boxed links; true: colored links
    	linkcolor=blue,          	% color of internal links
    	citecolor=blue,        		% color of links to bibliography
    	filecolor=magenta,      		% color of file links
		urlcolor=blue,
		bookmarksdepth=4
}
\makeatother
% ---


% ---
% Estilo de capítulos
%
% \chapterstyle{pedersen} 
%\chapterstyle{lyhne} 
%\chapterstyle{madsen} 
\chapterstyle{veelo} 
%
% Veja outros estilos em:
% https://www.ctan.org/tex-archive/info/MemoirChapStyles
% ---

% para cabeçalhos sem estar em maiúsculas
%\nouppercaseheads 

% -----
% Declarações de cabecalhos 
% -----
% Cabecalho padrao
\makepagestyle{abntbookheadings}
\makeevenhead{abntbookheadings}{\ABNTEXfontereduzida\thepage}{}{\ABNTEXfontereduzida\textit\leftmark}
\makeoddhead{abntbookheadings}{\ABNTEXfontereduzida\textit\rightmark}{}{\ABNTEXfontereduzida\thepage}
\makeheadrule{abntbookheadings}{\textwidth}{\normalrulethickness}

% Cabecalho do inicio do capitulo
\makepagestyle{abntbookchapfirst}
\makeoddhead{abntbookchapfirst}{}{}{}

% Configura layout para elementos textuais
\renewcommand{\textual}{%
  \pagestyle{abntbookheadings}%
  \aliaspagestyle{chapter}{abntbookchapfirst}% customizing chapter pagestyle
  \nouppercaseheads%
  \bookmarksetup{startatroot}% 
}
% ---

% ---
% Espaçamentos entre linhas e parágrafos
% ---
% O tamanho do parágrafo é dado por (exemplo):
%\setlength{\parindent}{1.3cm}
%% Não recomendado mudar.
 
% Controle do espaçamento entre um parágrafo e outro:
%\setlength{\parskip}{0.2cm}  % tente também \onelineskip
%% Não recomendado mudar.

% Margens do documento 
%% (margens do abntex2 não combinam nem com A5 nem com estilos de capítulo da
% classe memoir.)
\setlrmarginsandblock{2.5cm}{3.5cm}{*}
\setulmarginsandblock{2.5cm}{3.5cm}{*}
\checkandfixthelayout
% ---


% ---
% Início do documento
% ---
\begin{document}
\frenchspacing

\frontmatter

% ---
% Capa principal
% ---
\begin{titlingpage}
\phantom{xxx}
\vspace{0.5cm}
\huge
\raggedright
\imprimirautor\\
\vspace{2.5cm}
\huge 
{\raggedleft
\includegraphics[scale=0.9]{abntex2-modelo-img-marca.pdf}\\[1cm]
\textit{\textcolor{blue}{\imprimirtitulo}}\\[1cm]
}
\centering 
%  %este é um símbolo que só aparecerá com a fonte Minion.
\vfill
\Large
% %este é um símbolo que só aparecerá com a fonte Minion.
\imprimirinstituicao
\end{titlingpage}
% ---

% ---
% Contra-capa
% ---
\begin{titlingpage}

\phantom{xxx}
\vspace{0.5cm}
\huge
\raggedright
\imprimirautor\\
\vspace{2.5cm}
\huge 
{\raggedleft
\includegraphics[scale=0.9]{abntex2-modelo-img-marca.pdf}\\[1cm]
\textit{\textcolor{blue}{\imprimirtitulo}}\\[1cm]
}
\centering 
% %este é um símbolo que só aparecerá com a fonte Minion.
\vfill
\Large
% %este é um símbolo que só aparecerá com a fonte Minion.
\imprimirinstituicao
% ---

% ---
% Verso da contra-capa
% ---
\clearpage
\ABNTEXfontereduzida
%\raggedright
© 2017 \imprimirautor \space \& \imprimirinstituicao
%este é só um exemplo de copyright.

Qualquer parte deste livro pode ser reproduzida, desde que citada a fonte.

\vspace*{\fill}

\begin{center}
Dados Internacionais de Catalogação na Publicação (\textsc{cip})
Câmara Brasileira do Livro, \textsc{sp}, Brasil
\end{center}

\begin{mdframed}
\noindent MELO, Gabriel M. de.

\imprimirtitulo. / \imprimirautor. -- \imprimirlocal: \imprimirinstituicao
Ltda., 2017.

\medskip

Bibliografia.

ISBN XXXX-XXXX-XX.

\medskip

1. Programas de computador. 2. Tipografia. 3. Latex. 4. Normas ABNT.

\end{mdframed}

\end{titlingpage}
% ---

% ---
% Agradecimentos
% ---
\begin{agradecimentos}
Este livro é fruto da soma de esforços mútuos do núcleo de estudos \textit{EMakers}\footnote{emakers.com} em integrar e disseminar o conhecimento na área de Sistemas Embarcados, Otimização e Internet das Coisas para o estudantes e entusiastas brasileiros. Um agradecimento especial ao professor - e coordenador - Bruno de Abreu Silva que, pacientemente, vem nos auxiliando e orientando durante nossos projetos e almejos. Agradeço também ao Departamento de Ciência da Computação da UFLA, que, com portas abertas, sempre nos incentiva e motiva a trabalhar.
\end{agradecimentos}
% ---

% ---
% inserir lista de ilustrações
% ---
\pdfbookmark[0]{\listfigurename}{lof}
\listoffigures*
\cleardoublepage

% ---
% inserir lista de tabelas
% ---
\pdfbookmark[0]{\listtablename}{lot}
\listoftables*
\cleardoublepage
% ---

% ---
% inserir o sumario
% ---
\pdfbookmark[0]{\contentsname}{toc}
\tableofcontents*
\cleardoublepage
% ---

% ------------------------------------------------------------
% Início da parte textual
% ------------------------------------------------------------
%\textual
\mainmatter
% ------------------------------------------------------------

% ------------------------------------------------------------
\chapter*[Introdução]{Introdução}
\addcontentsline{toc}{chapter}{Introdução}
% ------------------------------------------------------------

\lettrine[nindent=0.35em,lhang=0.40,loversize=0.3]{E}{ste livro} dá início a uma
série de documentos didáticos voltados à eletrônica digital e analógica, programação e gestão produzidos pelo grupo EMakers na Universidade Federal de Lavras. 

O livro foi estruturado em 5 capítulos, subdivididos em seções que aprofundam uma temática, contendo trecho de códigos e/ou tabelas.

O primeiro capítulo, denominado \textsf{\textbf{Contextualizando}}, é destinado à introduzir o leitor aos principais conceitos necessários para a compreesão básica do restante do livro. São vistos contextos históricos e uma breve revisão sobre eletrônica digital básica.

O capítulo seguinte, \textsf{\textbf{O ESP8266}}, apresenta o microcontrolador tema da obra e seus principais aspectos de \textit{hardware}, como \textit{GPIOs}, interfaces de comunicação, protocolos, memória, consumo e demais componentes.



Algumas notações são padronizadas para referir termos repetitivos no decorrer do livro:



%Em geral, qualquer classe do \LaTeX\ que contemple o formato de livros poderia
%ser utilizada (como \textsf{book}, \textsf{memoir} e \textsf{scrbook}, entre
%outras). A formatação de geral dos capítulos, margens, tamanho de página,
%fontes, etc., segundo a norma ABNT em questão, pode ser modificada pelo usuário
%à vontade.

%Porém, este modelo foi composto com a classe \textsf{abntex2.cls} com o intuito
%de estimular que autores de teses e dissertações convertam e publiquem seus
%trabalhos em forma de livro. 

%Essencialmente, este modelo é idêntico aos demais modelos distribuídos com o
%\abnTeX. Porém, este documento exemplifica como customizar a formatação final do
%documento para que ele fique adequado aos padrões de publicação de livros.
%Observe que há vários comentários no código-fonte do arquivo, de modo a
%facilitar ao máximo as customizações. Consulte o portal do projeto para obter
%acesso aos manuais, wiki, e grupos de discussões do \abnTeX.

%Em linhas gerais, a norma ABNT NBR 6029:2006 não estabelece parâmetros tão
%rígidos quanto a ABNT NBR 14724:2011, para trabalhos acadêmicos, e segue, de
%certa forma, o design usual de livros\footnote{Para facilitação da compreensão
%de termos técnicos, ver a Tabela \ref{vocabulario-texto} com alguns termos do
%design de livros.}.

%Desta forma, temos:

%\begin{description}
%\item[parte pré-textual] Composta por: falsa folha de rosto, folha de rosto,
%colofão (opcional na parte pós-textual), sumário (conteúdo),
%a\-gra\-de\-ci\-men\-tos (acknowledgments), dedicatória, epígrafe;
%\item[folha de rosto] Como diz o nome em inglês (\textit{title page}) a ``folha
%de rosto'' é a \textit{página do título}. No verso da folha de rosto, costuma-se
%incluir os dados sobre a obra e a edição, catalogação, editora, direitos
%autorais e de reprodução, etc;
%\item [parte pós-textual] Composta por epílogo, posfácio, apêndice, glossário,
%bibliografia, índice remissivo (inglês: \textit{index}), colofão, etc;

%\end{description}

%Este documento deve ser utilizado como complemento dos manuais do \abnTeX\ 
%\cite{abntex2classe,abntex2cite,abntex2cite-alf} e da classe \textsf{memoir}
%\cite{memoir}. 

% -------------------------
\chapter{Contextualizando}
% ------------------------------------------------------------
  
  \lettrine[nindent=0.35em,lhang=0.40,loversize=0.3]{O}{ mundo}, bem como suas ferramentas, adapta aos seus mais variados e históricos momentos. Das macros transformações da sociedade às micros, o homem se motivou pela busca do conhecimento do planeta, da vida e de si mesmo, fazendo-o ser humano. Desta forma, com o acúmulo de fatos observados e através da intercomunicação, o ser humano pôde compartilhar experiências, se informar.
  
  A informação e seus meios de produção, armazenamento, transmissão, acesso, segurança e  uso foram drasticamente adulterados com o advento do mais poderoso meio de comunicação, a \textit{Internet}. Em menos de 30 anos, a \textit{Internet} impulsionou a aproximação - e a aculturação - de diferentes povos, línguas e estudos. Por uma perspectiva discente, a Internet democratizou, publicizou e agilizou o conhecimento. 

\section{O Termo \textit{Internet of Things}}

Um termo, consideravelmente recente, tomou espaço nas discussões de grandes empresas e organizações, a \textit{Internet of Things} (ou, simplesmente, \textit{IoT}), vem gerando uma mudança na percepção do ambiente, onde o movimento M2M é uma realidade e os elementos comuns do dia-a-dia comunicam entre si e são capazes de tomar decisões mais completas. Essa vertente vem de encontro com sistemas de automação, embarcados e até otimização do tempo e organização pessoal.

Este livro vem como um introdutório compilado de - \textit{traduzidas} - experiências referentes à uma plataforma didática prática, robusta e ideal para entusiasmados com o \textit{IoT} baseada em um microcontrolador chinês : o \textit{\textbf{ESP8266}}.

\section{Microcontroladores}


 
% ------------------------------------------------------------
\chapter{O ESP8266}
% ------------------------------------------------------------

\lettrine[nindent=0.35em,lhang=0.40,loversize=0.3]{O}{ ESP8266} é um microcontrolador de 32-bits com \textit{Wi-Fi} integrado desenvolvido pela \textit{Espressif Systems} surgido, em meados de 2014, para suprir a contínua demanda por uma plataforma que fosse de \textsf{\textbf{baixo consumo}} energético, \textsf{\textbf{compacta}} e de \textsf{\textbf{desempenho confiável}} na industria de \textit{IoT}.

Assim como a maioria dos modelos Arduino, o ESP possui GPIOs (Pinos de entrada e saída de propósito geral) e suporte a PWM (Modulação por largura de pulso). O upload do firmware é feito também pela UART (RX/TX), porém o ESP8266 conta com upload OTA (over-the-air), que é a gravação através de uma rede.  

\section{Modem \textit{Wi-Fi}}
Com um completo e autônomo sistema de conexão \textit{Wi-Fi}, o ESP8266 pode tanto desempenhar uma aplicação standalone, em modo AP, ou um se conectar a um \textit{host}, em modo STA\footnote{Conta ainda com modo AP\_STA}. Pode ainda ser aplicado a qualquer microcontrolador como um adaptador WiFi através de interfaces \textit{SPI}, \textit{I2C} e \textit{UART}. 
(suporte a 802.11 b/g/n/e/i)
\section{GPIOS}

ESP8266 possui 17 pinos GPIO que podem assumir varias funções através da devida programação de seus registradores. Cada um dos pinos pode ser configurado com pull-up ou pull-down interno e também alta impedância. Possui um pino ADC, TOUT(9), de resolução de 1024 bits, porém sua alimentação deve ser limitada de 0 a 1V.

\section{\textit{I2C}}
    \textit{I2C} (\textit{Inter-integrated Circuit}) é um protocolo de comunicação entre dispositivos que baseia-se em um barramento de apenas duas vias: SDA (Serial Data) por onde são transmitidos e recebidos os dados e SCL(Serial Clock) que dita a temporização do tráfego das informações. O grande diferencial do protocolo é que são permitidos, teoricamente, até 127 dispositivos distintos comunicando através de um mesmo barramento.
    O ESP8266 possui suporte a interface \textit{I2C}, tanto como Master quanto Slave, para comunicação com outros microcontroladores e outros equipamentos periféricos, como sensores. A pinagem do barramento é vista pela tabela abaixo:


\begin{table}[ht]
\centering
\begin{tabular}{c c l}
   \toprule
    Pino    & Função & Descrição  \\ [0.5ex]
    \midrule \midrule
    GPIO02(14)      & SDA   & Pino de dados   \\
    MTMS(9)          & SCL   & Pino de Clock            \\[1ex]
    \bottomrule
    \end{tabular}
     \caption{I2C\label{I2C-table}}
\end{table}
               
\section{Interrupção externa}
        Interrupções são eventos ou condições que levam o microcontrolador a pausar a execução de uma tarefa em andamento, executar outra temporariamente e, então, retornar para a tarefa inicial.
        Com exceção do pino GPIO16(8), todas as demais GPIOs do ESP8266 possuem funcionalidade de interrupção externa.
       \begin{lstlisting} 
void setup() {
    ...
    pinMode(pinoDeInterrupcao, INPUT);
    attachInterrupt(pinoDeInterrupcao, funcao, MODO);
    // MODO assume "RISING", "CHANGE" ou "FALLING"
    ...
    }
    
int funcao(){
    ...
}
    
        
        \end{lstlisting}

\section{Modos de Consumo de Energia}
O ESP8266 já possui modos de economia de energia* 
\begin{description}
\item[modem-sleep] Modo usado em aplicações que requerem a CPU funcionando, como em aplicações com PWM e I2S. "Desliga" o circuito do Modem Wi-Fi enquanto mantiver uma conexão Wi-Fi sem transmissão de dados. Consumo médio: 15mA.
\item[light-sleep] Durante o modo, a CPU pode ser suspendida em aplicações como uma interruptor Wi-Fi. Sem haver transmissão de dados, a o circuito do Modem Wi-Fi pode ser desligado e a CPU suspendida para economia de consumo de energia. Consumo médio: 0.9mA.
\item [deep-light] Durante o modo, Wi-Fi é totalmente desligado. Para aplicações onde existam longos intervalos de tempo sem transmissão de dados. Por exemplo, o monitoramento da temperatura ambiente, lendo dados durante um período, dormindo por outro período e acordando para reconectar a um ponto de acesso. Consumo médio: 20μA.

\end{description}
       


\chapter{\textit{Firmware}}

\lettrine[nindent=0.35em,lhang=0.40,loversize=0.3]{O}{ firmware} de fábrica traz, por padrão, o conjunto de instruções AT (PDF), porém é possível realizar o upload de outros firmwares para programação em linguagens mais comuns como o \textit{nodeMCU} (LUA) e o \textit{micropython} (Python). Ambos os firmwares baseam em um sistema interno de arquivos, com um arquivo "main" executado no boot e também possuem sua própria biblioteca que é atualizada por suas comunidades. 
Entretanto, a Arduino IDE possui, atualmente, suporte ao ESP8266, tornando possível a programação em Arduino (C++), utilizando, inclusive, algumas bibliotecas do mesmo sem quaisquer alterações.

\section{USB to UART converters}
        O ESP8266 não possui, nativamente, suporte para comunicação USB, porém existem conversores que facilitam a comunicação entre o microcontrolador e um PC para realização do upload do programa, por exemplo. 

É possível ainda utilizar de dispositivos que já possuam um conversor interno para realizar o upload, como o Arduino UNO. Interligando os pinos UART dos dois dispositivos (TX(ESP)->RX(Arduino) e RX(ESP)->TX(Arduino)* é possível realizar a comunicação do ESP8266 com um PC através da interface USB do Arduino.
*IMPORTANTE: vale lembrar que a faixa de tensão de operação do ESP8266 é de 2.5 a 3.6V, sendo necessário então um divisor de tensão do TX (Arduino) para o RX (ESP8266).O ESP8266.

\section{Primeiro programa}

% ------------------------------------------------------------
\chapter{ESP8266 {\normalsize{\emph{VS}}} Arduino}
% ------------------------------------------------------------




% ------------------------------------------------------------
\chapter{Exemplos}
% ------------------------------------------------------------



% ------------------------------------------------------------
\postextual % pós-textual
% ------------------------------------------------------------

% ------------------------------------------------------------
\bibliography{abntex2-modelo-references} % insere o arquivo de bibliografia
% ------------------------------------------------------------

% ------------------------------------------------------------
% Colofão: última página com informações sobre a composição do livro.
\cleardoublepage
\thispagestyle{empty} 

Sinta-se convidado a participar do projeto \abnTeX! Acesse o site do projeto em
\url{http://www.abntex.net.br/}. Também fique livre para conhecer, estudar,
alterar e redistribuir o trabalho do ABN\TeX, desde que os arquivos modificados
tenham seus nomes alterados e que os créditos sejam dados aos autores originais,
nos termos da ``The \LaTeX\ Project Public
License''\footnote{\url{http://www.latex-project.org/lppl.txt}}.

Encorajamos que sejam realizadas customizações específicas deste documento.
Porém, recomendamos que ao invés de se alterar diretamente os arquivos do
\abnTeX, distribua-se arquivos com as respectivas customizações.
Isso permite que futuras versões do \abnTeX~não se tornem automaticamente
incompatíveis com as customizações promovidas. Consulte
\citeonline{abntex2-wiki-como-customizar} par mais informações.


% ~\vfill Este texto foi composto em Minion Pro, de Robert Slimbach, e Myriad Pro,
% de Robert Slimbach e Carol Twombly.
~\vfill Este texto foi composto em Utopia, de Robert Slimbach, através do pacote \texttt{fournier}.

\end{document}
